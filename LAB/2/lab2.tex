\documentclass[12pt,a4paper]{article}
%\usepackage{ctex}

\usepackage{amsmath,amscd,amsbsy,amssymb,latexsym,url,bm,amsthm}
\usepackage{epsfig,graphicx,subfigure}
\usepackage{enumitem,balance}
\usepackage{wrapfig}
\usepackage{mathrsfs,euscript}
\usepackage[usenames]{xcolor}
\usepackage{hyperref}
\usepackage[vlined,ruled,commentsnumbered,linesnumbered]{algorithm2e}
\usepackage{listings}
\usepackage{color}

\definecolor{mygreen}{rgb}{0,0.6,0}
\definecolor{mygray}{rgb}{0.5,0.5,0.5}
\definecolor{mymauve}{rgb}{0.58,0,0.82}
\newtheorem{theorem}{Theorem}
\newtheorem{lemma}[theorem]{Lemma}
\newtheorem{proposition}[theorem]{Proposition}
\newtheorem{corollary}[theorem]{Corollary}
\newtheorem{exercise}{Exercise}
\newtheorem*{solution}{Solution}
\newtheorem{definition}{Definition}
\theoremstyle{definition}

\lstset{
    columns=fixed,
    frame=shadowbox,
    backgroundcolor=\color[RGB]{245,245,244},
    keywordstyle=\color[RGB]{40,40,255},
    numberstyle=\footnotesize\color{darkgray},
    commentstyle=\it\color[RGB]{0,96,96},
    stringstyle=\rmfamily\slshape\color[RGB]{128,0,0},
    showstringspaces=false,
    language=bash
}

%\numberwithin{equation}{section}
%\numberwithin{figure}{section}

\renewcommand{\thefootnote}{\fnsymbol{footnote}}

\newcommand{\postscript}[2]
 {\setlength{\epsfxsize}{#2\hsize}
  \centerline{\epsfbox{#1}}}

\renewcommand{\baselinestretch}{1.0}

\setlength{\oddsidemargin}{-0.365in}
\setlength{\evensidemargin}{-0.365in}
\setlength{\topmargin}{-0.3in}
\setlength{\headheight}{0in}
\setlength{\headsep}{0in}
\setlength{\textheight}{10.1in}
\setlength{\textwidth}{7in}
\makeatletter \renewenvironment{proof}[1][Proof] {\par\pushQED{\qed}\normalfont\topsep6\p@\@plus6\p@\relax\trivlist\item[\hskip\labelsep\bfseries#1\@addpunct{.}]\ignorespaces}{\popQED\endtrivlist\@endpefalse} \makeatother
\makeatletter
\renewenvironment{solution}[1][Solution] {\par\pushQED{\qed}\normalfont\topsep6\p@\@plus6\p@\relax\trivlist\item[\hskip\labelsep\bfseries#1\@addpunct{.}]\ignorespaces}{\popQED\endtrivlist\@endpefalse} \makeatother
\title{VE482 LAB2}
\author{Wu Jiayao 517370910257 }
\date{September 2019}

\begin{document}

\maketitle
\section{Minix 3}
\subsection{}
  Using pkgin.
  \begin{lstlisting}[language=sh]
      pkgin install
      pkgin update
      pkgin remove
  \end{lstlisting}
\subsection{}
  \textbf{ifconfig} is for checking network state, configuring, controlling, and querying TCP/IP network interface parameters. \\
  \textbf{adduser} is to add a new user of the system. \\ 
  \textbf{passwd} is to set/delete/check the password for certain user.
\section{}
\subsection{}
    \begin{lstlisting}
      ssh root@localhost
    \end{lstlisting}
\subsection{}
    The default port is 22.
    To change for 2222, do on Minix 3
    \begin{lstlisting}[language=sh]
        vi /usr/pkg/etc/ssh/sshd_config
    \end{lstlisting}
    and edit the configuration of "Port"
    do on Linux
    \begin{lstlisting}[language=sh]
        ssh root@localhost -p2222
    \end{lstlisting}
\subsection{}
  \begin{lstlisting}
      ls $HOME/.ssh
      # Output
      config id_rsa id_rsa.pub known_hosts
  \end{lstlisting}
  \textbf{config} is the config file of ssh. \\
  \textbf{id\_rsa} is the private key of ssh. \\
  \textbf{id\_rsa.pub} is the public key of ssh. \\
  \textbf{known\_hosts} is the ssh server that has logged before. \\
  Do on Linux:
  \begin{lstlisting}[language=sh]
      vi ~/.ssh/config
  \end{lstlisting}
  Add contents into \textbf{config} file
  \begin{lstlisting}[language=sh]
    Host minix3
    Hostname localhost
    Port 2222
    User root
  \end{lstlisting}
  Connect Minix3 on Linux by 
  \begin{lstlisting}[language=sh]
    ssh minix3
  \end{lstlisting}
\subsection{}
  First, ssh client generates a pair of key, then sends the key to ssh server. The ssh server adds the key to the list of authorized hosts. \\
  Next time when the ssh client wants to login, it sends the key again to the ssh server. The ssh server uses public key cryptography to verify the identity of the ssh client. If the client is verified, a encrypted connection is made.
  \begin{lstlisting}
      ssh-keygen -t rsa
      scp -P2222 ~/.ssh/id_rsa.pub root@localhost:/root
      ssh minix3
      # In minix3
      cat ~/id_psa.pub >> ~/.ssh/authorized_keys
      exit
      ssh minix3 -i ~/.ssh/id_rsa
  \end{lstlisting}
\section{}
\subsection{}
  \begin{lstlisting}[language=sh]
    #/bin/bash
  \end{lstlisting}
\subsection{}
  \textbf{sh} is the origin shell of Unix, while \textbf{bash,csh,zsh} are all expansions of \textbf{sh}, with different features.
\subsection{}
  \begin{lstlisting}[language=sh]
      #define variables, no spaces
      myVar=1 
      #Using ${} to access variables.
      echo ${myVar} #access
  \end{lstlisting}
\subsection{}
  \textbf{\$0} means argv[0] in C/C++, the name of the command.\\
  \textbf{\$1} means argv[1] in C/C++, the first argument of the command.\\
  \textbf{\$n} means argv[n] in C/C++, the nth argument of the command.\\
  \textbf{\$?} means the exit state of the last command.\\
  \textbf{\$!} means the process ID of the most recently executed background.\\
\subsection{}
  \begin{lstlisting}[language=sh]
    #define array
    array=(var1 var2 var3 var4)
    #access elements
    ${array[0]}
    #assign elements
    array[3]=varX
  \end{lstlisting}
\subsection{}
  \begin{lstlisting}[language=sh]
    #if
    if[ "$1" = "YES" ];then
      echo $1
    elif[ "$1" = "NO" ];then
      echo 123
    else
      echo 321
    fi
    #switch
    case $1 in
      1)
        echo 1 ;;
      2)
        echo 2 ;;
      *)
        echo 0 ;;
    esac
  \end{lstlisting}
\subsection{}
  \begin{lstlisting}[language=sh]
    #for-in
    for file in $(ls); do
      echo $file;
    done
    #for(C style)
    for((i=0;i<10;i++));do
      echo 12345
    done
  \end{lstlisting}
\subsection{}
  \begin{lstlisting}[language=sh]
    #while
    int=0
    while (($int < $1)); do
        let int=int+1
    done
  \end{lstlisting}
\subsection{}
  PS3 provides a custom prompt for the user to select a value.
  \begin{lstlisting}
  PS3="Select a food (1-4): "
  select i in t m mt exit
  do
    case $i in
      t) echo "Tea";;
      m) echo "Milk";;
      mt) echo "Milktea";;
      exit) exit;;
    esac
  done
  \end{lstlisting}
\subsection{}
  The \textbf{iconv} command is used to convert between different character encodings. As there are various encodings for different OS, if one file is to be used in more than one OS, \textbf{iconv} command is very important.
\subsection{}
  \textbf{\$\{\#temp\}} means the length of \$temp. \\
  \textbf{\$\{temp\%\%word\}} deletes \textbf{word} on the right of \$temp. \\
  \textbf{\$\{temp/pattern/string\}} replaces \textbf{pattern} with \textbf{string} in \textbf{\$\{\#temp\}}.
\subsection{}
  Regular expression is a sequence of characters that define a search pattern.
  \begin{lstlisting}[language=sh]
    find . -regextype sed -regex ".*/*dmp*"
  \end{lstlisting}
\section{Warning Part}
\subsection{}
  Sed, stream editor, is a Unix utility that parses and transforms text, using a simple, compact programming language. It is efficent
  \begin{lstlisting}
    sed [-Ealn] command [file ...]
  \end{lstlisting}
  Awk is a pattern-directed scanning and processing language.
  \begin{lstlisting}
    awk [ -F fs ] [ -v var=value ] [ 'prog' | -f progfile ] [file ... ]
  \end{lstlisting}
\subsection{}
  \begin{lstlisting}
    curl --silent 'http://aqicn.org/? \
    city=Shanghai&widgetscript&size=large' \
    | sed 's/^.*title\=\\"Moderate\\">/AQ: /g' \
    | sed "s/<.*$//g" | sed -n "3p"
    curl --silent 'http://aqicn.org/? \
    city=Shanghai&widgetscript&size=large' \
    | sed "s/.*style\='font-size:10px;'>/Temp: /g" \
    | sed 's/<.*$//g' | sed -n '3p'
  \end{lstlisting}
\subsection{}
  \begin{lstlisting}
    ifconfig ap1 | awk '/ether /{print $2}'
    #For macbook pro 
  \end{lstlisting}
\end{document}