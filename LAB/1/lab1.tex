\documentclass[12pt,a4paper]{article}
\usepackage[utf8]{inputenc}
\usepackage{minted}
\setminted{linenos,breaklines,tabsize=4,xleftmargin=1.5em}
%\usepackage{ctex}
\usepackage{amsmath,amscd,amsbsy,amssymb,latexsym,url,bm,amsthm}
\usepackage{epsfig,graphicx,subfigure}
\usepackage{enumitem,balance}
\usepackage{wrapfig}
\usepackage{mathrsfs,euscript}
\usepackage[usenames]{xcolor}
\usepackage{hyperref}
\usepackage[vlined,ruled,commentsnumbered,linesnumbered]{algorithm2e}
\usepackage{listings}
\usepackage{color}

\definecolor{mygreen}{rgb}{0,0.6,0}
\definecolor{mygray}{rgb}{0.5,0.5,0.5}
\definecolor{mymauve}{rgb}{0.58,0,0.82}

\newtheorem{theorem}{Theorem}
\newtheorem{lemma}[theorem]{Lemma}
\newtheorem{proposition}[theorem]{Proposition}
\newtheorem{corollary}[theorem]{Corollary}
\newtheorem{exercise}{Exercise}
\newtheorem*{solution}{Solution}
\newtheorem{definition}{Definition}
\theoremstyle{definition}
\lstset{ 
  backgroundcolor=\color{white},   % choose the background color; you must add \usepackage{color} or \usepackage{xcolor}; should come as last argument
  basicstyle=\footnotesize,        % the size of the fonts that are used for the code
  breakatwhitespace=false,         % sets if automatic breaks should only happen at whitespace
  breaklines=true,                 % sets automatic line breaking
  captionpos=b,                    % sets the caption-position to bottom
  commentstyle=\color{mygreen},    % comment style
  deletekeywords={...},            % if you want to delete keywords from the given language
  escapeinside={\%*}{*)},          % if you want to add LaTeX within your code
  extendedchars=true,              % lets you use non-ASCII characters; for 8-bits encodings only, does not work with UTF-8
  firstnumber=1,                % start line enumeration with line 1000
%   frame=single,	                   % adds a frame around the code
  keepspaces=true,                 % keeps spaces in text, useful for keeping indentation of code (possibly needs columns=flexible)
  keywordstyle=\color{blue},       % keyword style
  language=Octave,                 % the language of the code
  morekeywords={*,...},            % if you want to add more keywords to the set
  numbers=left,                    % where to put the line-numbers; possible values are (none, left, right)
  numbersep=5pt,                   % how far the line-numbers are from the code
  numberstyle=\tiny\color{mygray}, % the style that is used for the line-numbers
  rulecolor=\color{black},         % if not set, the frame-color may be changed on line-breaks within not-black text (e.g. comments (green here))
  showspaces=false,                % show spaces everywhere adding particular underscores; it overrides 'showstringspaces'
  showstringspaces=false,          % underline spaces within strings only
  showtabs=false,                  % show tabs within strings adding particular underscores
  stepnumber=2,                    % the step between two line-numbers. If it's 1, each line will be numbered
  stringstyle=\color{mymauve},     % string literal style
  tabsize=2,	                   % sets default tabsize to 2 spaces
  title=\lstname                   % show the filename of files included with \lstinputlisting; also try caption instead of title
}

%\numberwithin{equation}{section}
%\numberwithin{figure}{section}

\renewcommand{\thefootnote}{\fnsymbol{footnote}}

\newcommand{\postscript}[2]
 {\setlength{\epsfxsize}{#2\hsize}
  \centerline{\epsfbox{#1}}}

\renewcommand{\baselinestretch}{1.0}

\setlength{\oddsidemargin}{-0.365in}
\setlength{\evensidemargin}{-0.365in}
\setlength{\topmargin}{-0.3in}
\setlength{\headheight}{0in}
\setlength{\headsep}{0in}
\setlength{\textheight}{10.1in}
\setlength{\textwidth}{7in}
\makeatletter \renewenvironment{proof}[1][Proof] {\par\pushQED{\qed}\normalfont\topsep6\p@\@plus6\p@\relax\trivlist\item[\hskip\labelsep\bfseries#1\@addpunct{.}]\ignorespaces}{\popQED\endtrivlist\@endpefalse} \makeatother
\makeatletter
\renewenvironment{solution}[1][Solution] {\par\pushQED{\qed}\normalfont\topsep6\p@\@plus6\p@\relax\trivlist\item[\hskip\labelsep\bfseries#1\@addpunct{.}]\ignorespaces}{\popQED\endtrivlist\@endpefalse} \makeatother
\title{VE482 LAB1}
\author{Wu Jiayao 517370910257 }
\date{September 2019}
\begin{document}
\maketitle
\section{Hardware overview}
\subsection{Where is the CPU hidden, and why?}
    \par CPU is hidden on the motherboard under the fan. The reason is to make sure the fan can efficiently cool CPU down, as CPU can produce much heat when working.
\subsection{What are the North and South bridges?}
    \par North bridge is a chip that deals with communications with CPU(through front-side bus), RAM and PCI Express, and the south bridge. It does the tasks that require higher performance. It is directly connected to CPU. It is also called Memory Controller Hub.
    \par South bridge is a chip that handles all of a computer's I/O functions. It implements slower capabilities of the motherboard. It is not directly connected to CPU. It is also called I/O Controller Hub.
\subsection{How are the North and South bridges connected together?}
    \par They are connected together through PCI bus. For example, for Intel, it's Direct Media Interface.
\subsection{What is the BIOS?}
    \par It is the short for Basic Input Ouput System. It is a firmware used to perform hardware initialization during booting process.
\subsection{Take out the CPU, rotate it and try to plug it back in a different position, is that working?}
    \par No. Some CPUs are designed so that they can only be plugged in one certain direction.
\subsection{Explain what overclocking is?}
    \par Overclocking is the action of increasing the clock rate of a certain component in the computer to exceed the rate certified by its manufacturer.
\subsection{What are pins on a PCI/PCI-e card and what are they used for?}
    \par Pins are tiny metail sticks at the bottom of PCI/PCI-e card. They are either used to connect ground, supply power to PCI/PCI-e card, transfer signal between motherboard and PCI-e card, or may be pulled low or sensed by multiple cards(Open drain), or tied together on card(Sense pin).
\newpage
\subsection{Before PCI-e became a common standard many graphics cards were using Accelerated Graphics
Port (AGP), explain why.}
    \par AGP is based on PCI. It provides connection between the slot and the processor instead of using the PCI bus, which allows higher clock rate. When PCI-e was released, since it performs better than PCI, AGP is replaced by PCI-e.
    
\section{Basic shell}
\subsection{Use the mkdir, touch, mv, cp, and ls commands to:}
\begin{lstlisting}[language=sh]
    # Create a file named test.
        touch test
        
    # Move test to dir/test.txt, where dir is a new directory.
        mkdir dir
        mv test dir/test.txt
        
    # Copy dir/test.txt to dir/test_copy.txt.
        cp dir/test.txt dir/text_copy.txt
        
    # List all the files contained in dir.
        ls dir -a
\end{lstlisting}
\subsection{Use the grep command to:}
\begin{lstlisting}[language=sh]
    # List all the files form /etc containing the pattern 127.0.0.1.
        grep -rl '127.0.0.1' /etc
    
    # Only print the lines containing your username and root in the file /etc/passwd (only one grep
should be used)
        grep -rE '(makersmelx|root)' /etc/passwd
\end{lstlisting}
\subsection{Use the find command to:}
\begin{lstlisting}[language=sh]
    # List all the files from /etc that have been accessed less than 24 hours ago.
        find /etc -atime 1
        
    # List all the files from /etc whose name contains the pattern “netw”.
        find /etc -name '*netw*'
\end{lstlisting}

\subsection{In the bash man-page read the part related to redirections. Explain the following signs \mintinline{c}{>}, \mintinline{c}{>>}, \mintinline{c}{<<<}, \mintinline{c}{>&1}, and \mintinline{c}{2>&1 >}. What is the use of the tee command.}
    \par \mintinline{c}{>} redirects the standard output into a file. (name written on the right) \\
    \mintinline{c}{>>} redirects and appends the standard output into a file. (name written on the right) \\
    \mintinline{c}{<<<} redirects the contents on the right as the standard input of the command on the left. \\
    \mintinline{c}{>&1} redirects the standard output into standard output. \\
    \mintinline{c}{2>&1 >}redirects the standard error into standard output, then redirects the origin standard output into a file. \\
\subsection{Explain the behaviour of the \mintinline{c}{xargs} command and of the $|$ sign.}
    \mintinline{c}{xargs} is used to build and execute commands from standard input. \\
    The $|$ sign pipes the standard output of the command on the left into the command on the right as its standard input.
\subsection{What are the \mintinline{c}{head} and \mintinline{c}{tail} commands? How to “live display” a file as new lines are appended?}
    \mintinline{c}{head} and \mintinline{c}{tail} are used to get the first several and last several lines of a file. \\
    Use \mintinline{c}{tail} -f option.
\subsection{How to monitor the system using \mintinline{c}{ps}, \mintinline{c}{top}, \mintinline{c}{free}, \mintinline{c}{vmstat}?}
    \mintinline{c}{ps} is to monitor process. \\
    \mintinline{c}{top} is to dynamically monitor processes, CPU and RAM. \\
    \mintinline{c}{free} is to monitor RAM. \\
    \mintinline{c}{vmstat} collects and displays summary information about operating system memory, processes, interrupts, paging and block I/O.
\end{document}
